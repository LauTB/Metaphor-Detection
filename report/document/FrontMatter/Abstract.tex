\begin{resumen}
El lenguaje ha presentado cambios desde que fue creado, no es el mismo español en que hablamos aquí y ahora en el país, que el español hablado hace siglos por nuestros ancestros. Los lingüistas se dedican a estudiar precisamente estos cambios en el idioma, ya que al ser una invención humana no es solamente una parte esencial de nuestras vidas, es también un resultado de nuestra cultura y horizonte histórico-cultural. Contribuyendo al proyecto CORESPUC se presenta este trabajo cuyo objetivo es ofrecer una herramienta al lingüista, la cual consistirá en una detección automática de la metáfora, la cual es una figura retórica que nace completamente del arte y cultura de las sociedades. Para el caso del idioma español, hay escasos avances en ese aspecto y con textos cubanos la investigación en ese aspecto es nula. En este trabajo se propone un algoritmo para detectar metáforas léxicas de tipo III(adjetivo-sustantivo) utilizando redes neuronales. Se presentan dos tipos de modelos: uno LSTM (Long Short Term Memory por sus siglas en inglés) y uno GRU(Gated Recurrent Unit) que son tipos de redes neuronales diseñadas para evitar el problema del desvanecimiento del gradiente y son particularmente buenas para trabajar con texto y secuencias. Utilizando uno de estos modelos, se construyó un corpus con frases extraídas del periódico villaclareño Vanguardia. Dichos modelos fueron comparados con los resultados del estado del arte como forma de validación.
\end{resumen}

\begin{abstract}
	The language has undergone changes since it was created, it is not the same Spanish we speak here and now in the country, as the Spanish spoken centuries ago by our ancestors. Linguists are dedicated to studying precisely these changes in the language, since being a human invention it is not only an essential part of our lives, it is also a result of our culture and historical-cultural horizon. Contributing to the CORESPUC project, this work is presented whose objective is to offer a tool to the linguist, which will consist of an automatic detection of the metaphor, which is a rhetorical figure that is born entirely from the art and culture of societies. In the case of the Spanish language, there is little progress in this aspect and with Cuban texts the research in this aspect is nil. In this work an algorithm is proposed to detect type III lexical metaphors (adjective-noun) using neural networks. Two types of models are presented: one LSTM (Long Short Term Memory) and one GRU (Gated Recurrent Unit) which are types of neural networks designed to avoid the problem of gradient fading and are particularly good to work with. text and sequences. Using one of these models, a corpus was built with sentences taken from the Villa Clara newspaper Vanguardia. These models were compared with the results of the state of the art as a form of validation.
\end{abstract}