\chapter{Estado del Arte}\label{chapter:state-of-the-art}
\section{Teoría de la metáfora}
Usualmente las metáforas son clasificadas en dos tipos: las conceptuales y las lingüísticas. Las primeras se basan en asociar conceptos otorgando atributos y propiedades de un concepto a otro, por ejemplo \textit{perder el tiempo} que, como explica [\cite{Lakoff}], hace referencia a el TIEMPO es DINERO dándole las propiedades concretas del dominio fuente(dinero) a un dominio objetivo más abstracto(tiempo). Las metáforas lingüísticas basan su carga metafórica en la estructura de los términos que utilizan.\\
Las metáforas conceptuales se clasifican por dos criterios: el grado de aceptación y el dominio del mapeo. \\
El grado de aceptación fue definido por [\cite{Nunberg}] y se basa en el tiempo que lleva existiendo dicha metáfora. Según este criterio tienen tres categorías:
\begin{itemize}
\item Metáfora novedosa o novel: Como su nombre indica, representa un mapeo innovador, algo que no se ha realizado antes. Un ejemplo de esto podría ser cuando surgieron las redes sociales y se utilizaban frases como \textit{debutó en Facebook} que crea la asociación entre el concepto de actuación y el concepto de las redes sociales.
\item Metáfora convencional: El uso de este tipo de mapeo está más esparcido. Es usual encontrar ejemplos en diccionarios, como sería \textit{perder el tiempo} que asocia al concepto \textit{tiempo} con el concepto \textit{dinero}.
\item Metáfora muerta: Este tipo perdió el contexto donde era válida con el paso del tiempo. Pueden encontrarse en textos literarios antiguos.
\end{itemize}
El dominio de mapeo fue descrito y recopilado en los textos \textit{Metaphors we live by} y \textit{Master metaphor list} por [\cite{LakoffetAl}] y [\cite{Lakoff}] respectivamente. Según el dominio del mapeo se tienen las siguientes categorías:
\begin{itemize}
\item Metáforas estructurales. Las características del dominio objetivo se expresan en función del dominio fuente mediante asociaciones. Un ejemplo es la asociación entre el concepto \textit{expresiones} con el concepto \textit{contenedores}, se puede apreciar en frases como \textit{palabras vacías}.
\item Metáforas de orientación Pone un Sistema de conocimientos en función de otro generalmente asociado a la orientación espacial. Un ejemplo de esto es \textit{caer en el alcoholismo} donde se asocia el concepto de abajo con un concepto negativo llevando a la idea de que \textit{abajo es malo}.
\item Metáforas ontológicas: Discretizan y cuantifican cosas continuas y no cuantificables como las emociones al atribuir características humanas a algo. Ejemplo: \textit{la inflación avanza} le asocia al concepto \textit{inflación} la capacidad de movimiento que comúnmente tienen los humanos.
\end{itemize}

Por otra parte, las metáforas lingüísticas se clasifican función de la cantidad palabras que llevan la carga metafórica y se dividen en tres categorías según la propuesta de [\cite{Sunny}]. 
\begin{itemize}
\item  Metáforas léxicas: La carga metafórica está en una sola palabra, existen varios subtipos:
\begin{enumerate}
\item Tipo I o Nominales. Se tienen dos sustantivos que se conectan por un verbo copulativo y el mapeo se realiza de forma explícita.
\item Tipo II o Sujeto -Verbo - Objeto (SVO). En este caso el mapeo ocurre de forma implícita y el verbo tiene la carga metafórica.
\item Tipo III o Adjetivo -Sustantivo (AN por sus siglas en inglés): AL igual que en el tipo anterior el mapeo es implícito. La diferencia es que en este tipo la carga metafórica la lleva el adjetivo 
\item Tipo IV o Adverbio-Verbo (AV): Sucede lo mismo que con el tipo anterior solo que en este caso el adverbio es el encargado de llevar la carga metafórica.
\end{enumerate}
\item Metáforas multi-palabra: En este caso son dos o más palabras tienen la carga metafórica como una sola unidad léxica Un ejemplo de esto se ve en las frases verbales.
\item Metáforas extendidas: Poseen amplias analogías con mucha complejidad. Su uso es casi exclusivo de la literatura.
\end{itemize}

\section{Enfoques computacionales a la metáfora}
A medida que ha ido avanzando la ciencia se han presentado diferentes enfoques a la metáfora. \\
Uno de los primeros en surgir fue el \textbf{Enfoque de sustitución} el cual se basa en ver la metáfora como forma de comunicarse cuando no hay algo mejor. Sus basamentos están en la lógica proposicional en la que una expresión es verdadera o falsa en un determinado contexto, en el caso de la metáfora esta aparecería como una incongruencia en el escenario. \\
Extendiendo el enfoque anterior, surge el \textbf{Enfoque por comparación} que añade la idea de ver la metáfora como un símil implícito. Ambos enfoques fueron severamente criticados por simplificar demasiado.\\
Existen enfoques que intentan escapar de las dificultades de los mencionados anteriormente, como el \textbf{Enfoque de violación a la preferencia de selección} que plantea que las unidades léxicas ocurren acompañadas de otras dando lugar a patrones determinados por la convergencia de características sintácticas, esto se conoce como principio de preferencia de selección y se dice que se encuentra una metáfora. Por otra parte, está el \textbf{Enfoque de inclusión de clase} que se especializa en metáforas del tipo \textit{X es Y} basándose en que la comparación entre el dominio X
y el dominio Y pueda tratarse de una metáfora si X pertenece a la superclase de Y.\\
Otros enfoques intentan analizar la metáfora como un fenómeno cognitivo propio. Entre ellos está el \textbf{Enfoque de interacción} que no se basa en definir dominio de partida y de llegada. se enfoca en un termino consentido.Ese término es el enfoque metafórico y el contexto que lo rodea es el macro literal.Los participantes en la interacción que estén familiarizados con sentido literal y tengan un trasfondo cultural similar con respecto al enfoque metafórico tienen las herramientas para interpretar la metáfora. El mapeo sale de la interacción con la metáfora en lugar de ser algo predefinido por lo que las metáforas noveles son fáciles de interpretar según la experiencia y conocimiento previo.
Similar a la idea anterior se encuentra el \textbf{Enfoque de mapeo conceptual} que también ve la metáfora como proceso cognitivo, la diferencia es que este presta atención a los conceptos en lugar de a las palabras de una expresión metafórica. Se basa en la re conceptualización de un dominio más abstracto(salida) a uno más concreto (llegada). Los mapeos conceptuales resultantes son supuestamente estables pero las metáforas lingüísticas pueden ser mapeadas de incontables formas. Sus detractores critican la falta de hipótesis que conceptualice el proceso de re conceptualización especialmente cuando hay múltiples mapeos conceptuales.\\
Finalmente se tiene el \textbf{Enfoque basado en corpus} que está basado en análisis estadístico de una gran cantidad de datos para examinar la ocurrencia de fenómenos lingüísticos.Dependen de las herramientas y del entorno de la metáfora en cuestión. Los corpus suelen hacerse específicos del dominio.\\
Para este enfoque hay diversas técnicas, comúnmente implican matchear una serie de palabras (frases en el texto) y marcarlas con su contexto inmediato. Otra idea es comprobar las colocaciones que incluyan una palabra objetivo, por ejemplo: para buscar en un mapeo conceptual buscar un término relacionado a un tema específico; la principal desventaja es que alguien tiene que asignar dominios.

\section{Procesamiento computacional y la metáfora}
La investigación sobre la metáfora se divide en tres objetivos principales: interpretación, generación y detección.\\
La interpretación de la metáfora, como su nombre indica, busca desarrollar un programa que reciba una metáfora como entrada y devuelva lo que significa.La dificultad está en que la maquina entienda el significado de la metáfora y lo traduzca a una forma no ambigua. Requiere que la máquina posea un amplio conocimiento amplio del mundo y de cultura.\\
La generación automática de metáforas tiene las mismas dificultades que la anterior. Añadido a eso, se espera que la computadora tenga creatividad y coherencia. Es la menos estudiada.\\
La detección de metáforas se puede ver como un problema de etiquetado donde cada token tiene una etiqueta con un conjunto predeterminado de etiquetas. También puede verse como un problema de clasificación que tiene por objetivo es decidir si una declaración es de la categoría literal o metafórica basado en un set de propiedades aprendidas.\\
Al principio se usaban buscan reglas codeadas a mano y se buscaban infracciones de las mismas con preferencia de selección [\cite{Fass}], apoyados en recursos como Word Net [\cite{Manson}]. Los métodos estadísticos usaban métricas del corpus: frecuencia de palabras, cálculo de similitud, clustering de oraciones para desambiguar el uso de verbos. \\
En 2014 [\cite{Tsvetkov}] creó un corpus con metáforas léxicas de tipo II y III con el uso de un Random Forest. Adicionalmente incluyeron expresiones con la misma estructura que las metáforas pero que no tuvieran carga metáfórica.\\
Por otra parte, existe la tarea compartida de identificación de metáforas que le da un enfoque de etiquetado sequencial, [\cite{LeongEtAl}] redactó un reporte en 2018 sobre dicha tarea utilizando el corpus VUA. En ese mismo año [\cite{Opolka}] utilizó un modelo de redes neuronales sobre el corpus previamente mencionado con el uso de word embeddings [\cite{Mikolov}].\\
En el caso del idioma español, el progreso es más escaso. Se tienen los esfuerzos de [\cite{Santiago}] para anotar 306 oraciones con 286 sustantivos y verbos etiquetados como metáfora, lamentablemente este recurso no es de dominio público. Desde ese entonces existieron algunos trabajos para dominios específicos como textos académicos [\cite{Ureña}] entre otros. Finalmente en [\cite{Sánchez}]construyó un corpus en español de dominio público para metáforas léxicas conocido como CoMeta.





