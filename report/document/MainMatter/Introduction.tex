\chapter*{Introducción}\label{chapter:introduction}
\addcontentsline{toc}{chapter}{Introducción}
El lenguaje ha presentado cambios desde que fue creado, no es el mismo español en que hablamos aquí y ahora en el país, que el español hablado hace siglos por nuestros ancestros. Los lingüistas se dedican a estudiar precisamente estos cambios en el idioma, ya que al ser una invención humana no es solamente una parte esencial de nuestras vidas, es también un resultado de nuestra cultura y horizonte histórico-cultural.
Es por lo anterior que estudiar el lenguaje es importante, derivado de esta misma importancia nace un proyecto de la Facultad de Artes y Letras y la Facultad de Matemática y Computación de la Universidad de La Habana para la construcción y explotación de un corpus del español público de Cuba (CORESPUC).\\
Contribuyendo al proyecto se presenta este trabajo cuyo objetivo es ofrecer una herramienta al lingüista, la cual consistirá en una detección automática de la metáfora, la cual es una figura retórica que nace completamente del arte y cultura de las sociedades. \\
La principal motivación para tener una herramienta que detecte metáforas es poder crear un registro histórico de cómo han evolucionado las metáforas a lo largo del discurso político cubano, lo cual hasta el momento de la creación del software que se presenta no había sido posible. \\
El problema de detección de metáforas es uno de los tantos problemas de NLP(\textit{Natural Language Processing} por sus siglas en inglés) y consiste en dado un texto poder seleccionar la palabra o frase que no se esté utilizando en un sentido literal sino que sea una metáfora. Un ejemplo sería \textit{"el tiempo es dinero"} que sería una metáfora del tipo conceptual donde se asocia al concepto de tiempo el atributo de ser valioso como el dinero.\\ 
Con el creciente desarrollo de las técnicas de Aprendizaje Automático(Machine Learning o ML en inglés) es natural pensar que se usen para intentar resolver el problema que se presenta. En efecto así ha ocurrido con el desarrollo de corpus como el VUA [\cite{LeongEtAl}] que es uno de los más conocidos para la experimentación en el idioma inglés y el corpus creado por [\cite{Tsvetkov}] que fue utilizado como base en el desarrollo de esta investigación. Lamentablemente la detección de metáforas para el idioma español no se encuentra tan estudiada como en el idioma inglés pero si se cuenta con un corpus llamado CoMeta [\cite{Sánchez}] que fue utilizado como comparativa en el desarrollo de esta tesis.\\
El principal aporte logrado es un modelo entrenado capaz de reconocer metáforas en textos de autoría cubana, esto servirá para futuras investigaciones sobre el desarrollo del lenguaje en el país. También cabe destacar que con el desarrollo de la detección de metáforas es posible mejorar la calidad de los resultados de otros problemas como la traducción automática, desambiguar usos de palabras e incluso detección de ironía o sarcasmo en el análisis de sentimientos.\\

La hipótesis que se quiere probar es si el modelo de red neuronal propuesto y entrenado con el corpus creado por [\cite{Tsvetkov}] es capaz de reconocer metáforas en español vía traducción automática.\\
El problema que resuelve esta tesis es la construcción de un modelo de redes neuronales capaz de reconocer metáforas en textos del periódico Vanguardia. El objeto de investigación es las redes neuronales para procesamiento de texto y el campo de acción es el procesamiento del lenguaje natural.\\
 El objetivo general es el diseño e implementación de un algoritmo para el estudio de la metáfora en el periódico digital Vanguardia. Como objetivos específicos se tienen los siguientes:
\begin{itemize}
\item Construir el corpus del periódico Vanguardia anotando etiquetas morfosintácticas y entidades nombradas: Lo primero que se necesita es obtener el texto que se va a utilizar tanto como para entrenar como para probar los resultados del modelo. Para esto se utilizaron técnicas de raspado de la web.
\item Verificar la aparición o no de noticias entre la versión PDF y la versión online del periódico: Se han dado casos donde ocurren ciertas diferencias entre ambas versiones(probablemente debido a errores humanos) por lo que se hace necesaria dicha verificación.
\item Proponer algoritmo para el estudio de la metáfora: Como se ha explicado el algoritmo propuesto consiste en entrenar. En secciones posteriores se explicará en profundidad como lograrlo.
\item Implementación de una interfaz gráfica para el uso de los lingüistas: Como se desea que el modelo sea utilizado por personas que no son científicos de la computación fue necesario crear una interfaz visual para mostrar los resultados que se obtuvieron.
\end{itemize}

En el capítulo 1 se describen los trabajos relacionados a este que conforman el estado del arte, el capítulo 2 explica en profundidad el modelo propuesto para la solución del problema desde el punto de vista teórico, en el capítulo 3 se dan detalles sobre la implementación tanto del modelo como del preprocesamiento necesario para los datos y también se muestran los resultados de la experimentación con el modelo, finalmente se presentan las conclusiones y las recomendaciones para futuras investigaciones. 